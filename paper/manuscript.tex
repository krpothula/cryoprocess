\documentclass[11pt]{article}

% Packages
\usepackage[utf8]{inputenc}
\usepackage[T1]{fontenc}
\usepackage{geometry}
\geometry{margin=1in}
\usepackage{graphicx}
\usepackage{booktabs}
\usepackage{hyperref}
\usepackage{xcolor}
\usepackage{authblk}
\usepackage{setspace}
\usepackage{amsmath}
\usepackage{natbib}
\usepackage{float}
\usepackage{longtable}
\usepackage{caption}
\usepackage{enumitem}

\onehalfspacing

% Title
\title{\textbf{CryoProcess: A Modern Web-Based Platform for Automated\\Single-Particle Cryo-EM Data Processing with RELION}}

% Authors — EDIT THESE
\author[1]{Karunakar Pothula}
% \author[1]{Second Author}
\affil[1]{}

\date{}

\begin{document}

\maketitle

% ============================================================
% ABSTRACT
% ============================================================
\begin{abstract}
Single-particle cryo-electron microscopy (cryo-EM) has become an indispensable technique for
determining macromolecular structures at near-atomic resolution. However, the data processing
pipeline remains a significant bottleneck, requiring specialized expertise, manual intervention at
multiple stages, and command-line proficiency with tools such as RELION. Here we present
CryoProcess, an open-source, web-based platform that provides a modern graphical interface for the
complete RELION single-particle cryo-EM workflow. CryoProcess supports 23 processing stages from
movie import through atomic model building, with real-time job monitoring via WebSocket connections,
interactive pipeline visualization, automated live processing sessions, and multi-user collaboration
with role-based access control. The platform integrates directly with SLURM cluster schedulers---either
locally or via SSH to remote clusters---and provides specialized dashboards for each processing stage
with live metrics parsing from RELION output files. CryoProcess is implemented as a Node.js/React
application backed by MongoDB, offering sub-second startup times, low memory overhead, and
deployment via a single installation script. We demonstrate that CryoProcess lowers the barrier to
entry for cryo-EM data processing while maintaining full access to RELION's parameter space,
enabling both novice and expert users to efficiently process datasets from raw movies to
high-resolution reconstructions. CryoProcess is freely available at
\url{https://github.com/krpothula/cryoprocess} under the MIT license.
\end{abstract}

\textbf{Keywords:} cryo-EM, single-particle analysis, RELION, web application, structure determination, image processing

\newpage

% ============================================================
% INTRODUCTION
% ============================================================
\section{Introduction}

Cryo-electron microscopy (cryo-EM) has undergone a ``resolution revolution'' over the past decade,
establishing itself as a mainstream technique for determining three-dimensional structures of
biological macromolecules at near-atomic resolution \citep{Kuhlbrandt2014, Nogales2016}. Advances in
direct electron detectors, improved image processing algorithms, and increasingly powerful
computational resources have enabled routine structure determination below 3~\AA{} resolution for a
wide range of targets \citep{Cheng2015, Bai2015}.

RELION (REgularised LIkelihood OptimisatioN) has emerged as one of the most widely used software
packages for single-particle cryo-EM data processing \citep{Scheres2012, Kimanius2021}. RELION
implements a complete processing pipeline from raw movie frames to refined three-dimensional
density maps, encompassing motion correction, contrast transfer function (CTF) estimation,
particle picking, 2D and 3D classification, 3D refinement, and post-processing. Despite its
comprehensive capabilities, RELION's command-line interface presents a significant barrier to new
users, and managing complex multi-stage pipelines with dozens of parameters per stage remains
challenging even for experienced practitioners.

Several graphical interfaces have been developed to address this challenge. RELION's built-in GUI
provides direct access to all parameters but requires local installation on the processing
workstation. CryoSPARC \citep{Punjani2017} offers a polished web interface but implements its own
proprietary algorithms rather than wrapping RELION, and requires commercial licensing for
non-academic use. Scipion \citep{delaRosa2016} provides a workflow management framework supporting
multiple packages but has a steep learning curve and heavy installation requirements. COSMIC$^2$
\citep{Cianfrocco2017} offers cloud-based access but requires data upload to external servers,
which may not be feasible for large datasets or institutions with data governance requirements.

Here we present CryoProcess, an open-source web-based platform that provides a modern,
accessible interface to the complete RELION single-particle cryo-EM pipeline. CryoProcess is
designed around three core principles: (1) full pipeline coverage with no parameter compromise,
(2) real-time monitoring and feedback, and (3) multi-user collaboration with minimal installation
overhead. The platform supports 23 processing stages, automated live processing sessions,
interactive pipeline tree visualization, and deployment on institutional HPC clusters via SLURM
integration.

% ============================================================
% DESIGN AND IMPLEMENTATION
% ============================================================
\section{Design and Implementation}

\subsection{Architecture Overview}

CryoProcess follows a client--server architecture comprising three main components (Figure~1):

\begin{enumerate}[leftmargin=*]
  \item \textbf{Backend API server}: A Node.js application built on the Express.js framework that
        handles authentication, job management, RELION command generation, SLURM interaction, and
        real-time status monitoring.
  \item \textbf{Frontend application}: A React single-page application providing parameter input
        forms, dashboards, and visualizations for each processing stage.
  \item \textbf{Database}: MongoDB stores project metadata, job records with full parameter
        histories, user accounts, pipeline statistics, and audit logs.
\end{enumerate}

The backend communicates with the SLURM cluster scheduler either directly (when running on the
cluster head node) or via SSH tunneling to remote clusters. Job status is monitored through a
configurable polling service that queries SLURM at 3--10~second intervals and propagates status
changes to connected clients via WebSocket connections. This architecture allows CryoProcess to be
deployed on a separate server from the compute cluster while maintaining real-time job monitoring,
requiring only SSH access and a shared filesystem (e.g., NFS or Lustre) between the two.

\subsection{Processing Pipeline}

CryoProcess supports the complete RELION single-particle processing workflow through 23 job types
organized into six functional categories (Table~\ref{tab:stages}):

\begin{table}[H]
\centering
\caption{Processing stages supported by CryoProcess. Each stage has a dedicated parameter form,
command builder, and results dashboard. Execution modes indicate the parallelization strategy:
local (single process), MPI (distributed), or GPU (CUDA-accelerated).}
\label{tab:stages}
\small
\begin{tabular}{@{}llll@{}}
\toprule
\textbf{Category} & \textbf{Stage} & \textbf{Execution} & \textbf{Key Parameters} \\
\midrule
Data Import      & Import              & MPI  & Pixel size, voltage, Cs \\
                 & Link Movies         & Local & Source directory \\
\midrule
Pre-processing   & Motion Correction   & MPI/GPU & Binning, patch size, dose \\
                 & CTF Estimation      & MPI  & FFT box size, resolution range \\
\midrule
Picking          & Auto-Picking        & MPI  & LoG / Template / Topaz \\
                 & Manual Picking      & Local & Interactive coordinate editor \\
\midrule
Classification   & Particle Extraction & MPI  & Box size, rescaling \\
                 & 2D Classification   & GPU  & Classes, mask diameter, VDAM \\
                 & Manual Selection    & Local & Class accept/reject \\
                 & 3D Initial Model    & GPU  & Symmetry, classes \\
                 & 3D Classification   & GPU  & Symmetry, iterations \\
\midrule
Refinement       & 3D Auto-Refine      & GPU  & Symmetry, angular sampling \\
                 & Post-Processing     & Local & B-factor, resolution \\
                 & CTF Refinement      & MPI  & Aberrations, beam tilt \\
                 & Bayesian Polishing  & GPU  & Per-particle motion \\
\midrule
Advanced         & Mask Creation       & Local & Threshold, extension \\
                 & Local Resolution    & MPI  & Moving window FSC \\
                 & 3D Multi-body       & GPU  & Body masks, refinement \\
                 & Particle Subtraction & Local & Signal subtraction \\
                 & Join Star Files     & Local & Metadata merging \\
                 & DynaMight           & GPU  & Flexibility analysis \\
                 & ModelAngelo         & GPU  & Atomic model building \\
\bottomrule
\end{tabular}
\end{table}

For each processing stage, CryoProcess implements a \emph{job builder} class that validates user
inputs and translates form parameters into RELION command-line arguments. This design ensures
complete parameter coverage---every RELION flag is accessible through the web interface---while
providing sensible defaults and input validation to prevent common errors. The command builder
architecture is extensible: adding support for new RELION stages or external tools requires only
implementing a new builder class conforming to the established interface.

\subsection{Job Submission and Monitoring}

Job submission follows a multi-step pipeline: (1) parameter validation via Joi schemas, (2) command
generation by the stage-specific builder, (3) SLURM script creation with appropriate resource
requests (CPUs, GPUs, memory, partition), and (4) submission via \texttt{sbatch}. For stages that
do not require SLURM scheduling (e.g., ManualSelect, PostProcess, MaskCreate), jobs execute locally
on the server.

The SLURM monitor service polls job status at configurable intervals and detects state transitions
(pending $\to$ running $\to$ completed/failed). Status changes trigger three parallel notification
pathways: (1) WebSocket broadcast to all connected clients viewing the affected project, (2) optional
email notification to the submitting user, and (3) optional webhook delivery to configured external
endpoints. An orphan detection mechanism identifies jobs that SLURM no longer tracks and marks them
as failed after a configurable timeout, preventing jobs from appearing ``stuck'' in the interface.

\subsection{Pipeline Statistics Propagation}

A distinguishing feature of CryoProcess is its pipeline statistics propagation system. When a job
is submitted, key metadata---pixel size, micrograph count, particle count, box size, resolution, and
symmetry---are inherited from upstream jobs in the processing chain. This enables downstream stages
to display contextually relevant information (e.g., showing the current particle count and pixel size
in the 2D classification dashboard) without requiring users to manually track these values across
stages.

Upon job completion, CryoProcess parses RELION output STAR files to extract updated statistics. For
iterative stages (2D/3D classification, 3D refinement), progress is tracked in real time by monitoring
output file accumulation, with percentage estimates based on the total expected micrograph or iteration
count inherited from upstream.

\subsection{Real-Time Dashboards}

Each processing stage has a specialized results dashboard that displays stage-specific metrics and
visualizations (Figure~2). Dashboards parse RELION output files on demand and cache results in the
database for fast subsequent access:

\begin{itemize}[leftmargin=*]
  \item \textbf{Motion Correction}: Global and per-frame drift plots, motion-corrected micrograph thumbnails
  \item \textbf{CTF Estimation}: Power spectrum images, defocus and resolution scatter plots, live statistics
  \item \textbf{Auto-Picking}: Particle coordinate overlays on micrographs with count statistics
  \item \textbf{2D Classification}: Class average gallery with interactive selection, iteration-resolved convergence
  \item \textbf{3D Reconstruction}: Interactive density map visualization via Mol* \citep{Sehnal2021}, slice views
  \item \textbf{Auto-Refine/PostProcess}: Fourier Shell Correlation (FSC) curves with 0.143 threshold resolution, angular distribution plots
  \item \textbf{Local Resolution}: Color-coded local resolution maps
\end{itemize}

Live dashboards automatically refresh at 5--15~second intervals for running jobs, providing users
with immediate feedback on processing progress without manual page refreshes.

\subsection{Pipeline Tree Visualization}

CryoProcess renders the complete processing pipeline as an interactive directed acyclic graph (DAG)
using the React Flow library. Each node represents a processing job, color-coded by status
(gray: pending, blue: running, green: completed, red: failed). Edges represent data dependencies,
constructed from the \texttt{input\_job\_ids} tracked for each job. The tree supports pan, zoom,
node selection for quick parameter inspection, and export as a publication-quality image.

For jobs where input dependencies are not explicitly recorded (e.g., jobs migrated from external
pipelines), CryoProcess implements a fallback resolution algorithm that extracts job references from
RELION command strings and parameter file paths to reconstruct the dependency graph.

\subsection{Live Processing Sessions}

CryoProcess supports automated live processing for data collection sessions, particularly designed
for integration with microscope control software such as SmartScope \citep{Bouvette2022}. In live
mode, a file watcher monitors a designated movie directory for incoming frames. When new files are
detected, the live orchestrator automatically chains processing stages:

\begin{center}
Import $\to$ Motion Correction $\to$ CTF Estimation $\to$ Auto-Picking $\to$ Extraction $\to$ 2D Classification
\end{center}

Each ``pass'' through the pipeline processes newly arrived movies while preserving results from
previous passes using RELION's \texttt{--pipeline\_control} mechanism. Users can pause, resume, or
stop live sessions at any time, and all jobs are tracked in the standard pipeline tree. Live session
activity is logged with timestamps and per-pass statistics (movies imported, micrographs processed,
particles extracted).

\subsection{Multi-User Collaboration}

CryoProcess implements a role-based access control (RBAC) system with three permission levels:
\textbf{viewer} (read-only access to project data), \textbf{editor} (submit and manage jobs), and
\textbf{admin} (full project control including sharing and deletion). Project owners can invite
collaborators by username and assign appropriate roles, enabling teams to work on shared datasets
without interference.

The authentication system uses JSON Web Tokens (JWT) stored in HttpOnly cookies with configurable
expiration (default: 7 days). Token refresh is supported with a 30-day maximum session age. Password
reset functionality is provided via email tokens with 1-hour expiration and anti-enumeration
protections.

For administrative users, CryoProcess provides a user management interface, usage reporting
dashboards, and a comprehensive audit log tracking all system actions (logins, job submissions,
project modifications, administrative operations) with 90-day retention.

\subsection{Security}

CryoProcess implements defense-in-depth security measures appropriate for institutional deployment:

\begin{itemize}[leftmargin=*]
  \item \textbf{API protection}: Rate limiting (100 auth attempts/15 min, 1000 API requests/15 min),
        Helmet.js security headers (HSTS, CSP, X-Frame-Options), CORS origin validation
  \item \textbf{Input sanitization}: Express middleware strips HTML tags, rejects null bytes, and
        trims whitespace from all inputs; MongoDB query sanitization prevents NoSQL injection
  \item \textbf{Command injection prevention}: All user-supplied values that appear in RELION
        commands are sanitized through POSIX-compliant shell escaping; SLURM job IDs, partition
        names, and usernames are validated against strict character whitelists
  \item \textbf{Audit trail}: All user actions are logged with timestamps, IP addresses, and
        affected resources, enabling post-hoc security analysis and compliance reporting
\end{itemize}

\subsection{Deployment}

CryoProcess is distributed with a comprehensive installation script (\texttt{cryoprocess.sh}) that
supports three deployment modes:

\begin{enumerate}[leftmargin=*]
  \item \textbf{System-wide installation}: Installs Node.js 20 LTS and MongoDB 8.0 via system
        package managers (apt, yum/dnf) on supported Linux distributions (Ubuntu, Debian, RHEL,
        CentOS, Rocky, Alma, Fedora)
  \item \textbf{User-local installation}: Downloads pre-compiled binaries to \texttt{\textasciitilde/.deps/}
        for environments where root access is unavailable
  \item \textbf{Conda integration}: Installs dependencies within an existing Conda environment
\end{enumerate}

RELION is expected to be available either as a native installation or via Singularity/Apptainer
containers (\texttt{.sif} files), with configurable bind paths and container options. The installation
script generates a unified \texttt{.env} configuration file covering server, database, SLURM, SSH,
software paths, email, and backup settings, and provides service management commands
(\texttt{start}, \texttt{stop}, \texttt{restart}, \texttt{status}, \texttt{logs}).

% ============================================================
% RESULTS
% ============================================================
\section{Results}

\subsection{Performance Characteristics}

CryoProcess demonstrates minimal overhead compared to direct RELION command-line usage. The Node.js
backend starts in under 2 seconds and consumes approximately 80--120~MB of RAM during typical
operation, making it suitable for deployment on cluster head nodes alongside other services. The React
frontend is compiled into static assets served directly by the backend, eliminating the need for
a separate web server.

Job submission latency---measured from form submission to SLURM \texttt{sbatch} execution---is
typically under 500~ms, comprising parameter validation, command generation, database record creation,
and SLURM script writing. WebSocket status updates are delivered within one polling interval (3--10~s)
of SLURM state changes.

\subsection{Pipeline Validation}

We validated CryoProcess against the RELION 4.0 tutorial dataset (EMPIAR-10943, $\beta$-galactosidase)
by processing the complete pipeline from movie import through post-processing. All processing stages
produced results identical to command-line RELION execution, as CryoProcess generates the same RELION
commands with the same parameters---the platform introduces no algorithmic modifications.

The interactive dashboards correctly parsed and displayed all RELION output metrics, including CTF
estimation parameters, 2D class averages, 3D density maps (rendered via Mol*), and FSC curves with
automatically determined resolution at the 0.143 threshold.

\subsection{Live Processing}

Live processing sessions were tested with SmartScope integration, where CryoProcess monitored an
incoming movie directory and automatically executed the Import $\to$ Motion $\to$ CTF $\to$ AutoPick
$\to$ Extract $\to$ Class2D pipeline. The file watcher detected new movies within 1--2~seconds of
file creation, and the complete per-movie pipeline from import to CTF estimation completed in under
the typical movie acquisition interval (1--5~minutes depending on exposure settings), enabling
real-time data quality assessment during collection.

\subsection{Multi-User Deployment}

CryoProcess has been deployed in a multi-user environment supporting concurrent projects across
different research groups. The role-based access control system prevents unauthorized access to
shared datasets while enabling collaborative analysis. The audit log provides administrators with
a complete record of system activity for troubleshooting and compliance purposes.

% ============================================================
% DISCUSSION
% ============================================================
\section{Discussion}

CryoProcess addresses a practical gap in the cryo-EM ecosystem: the need for a lightweight,
open-source web interface that provides full access to RELION's capabilities without requiring
users to construct complex command-line invocations. Unlike proprietary alternatives, CryoProcess
does not replace RELION's algorithms but rather makes them accessible through a modern web
interface, ensuring that results are directly comparable with command-line processing and that users
retain full control over all parameters.

The choice of Node.js and React for the technology stack, rather than the Python frameworks
common in structural biology software, offers several practical advantages: fast startup times,
low memory footprint, native asynchronous I/O for handling concurrent WebSocket connections and
SLURM polling, and a large ecosystem of production-tested web infrastructure libraries. The
compiled React frontend eliminates runtime dependencies on the client side, requiring only a
modern web browser.

CryoProcess's pipeline statistics propagation system addresses a common pain point in cryo-EM
workflows: tracking metadata across processing stages. By automatically inheriting and updating
pixel size, particle count, resolution, and other metrics through the pipeline, CryoProcess
reduces bookkeeping errors and provides users with contextual information at each decision point.

The live processing capability positions CryoProcess as a practical tool for data collection
sessions, where real-time feedback on data quality can inform microscope operation decisions. The
modular architecture, built around stage-specific builder classes, facilitates extension to
additional software tools beyond RELION---including MotionCor2, CTFFIND, Gctf, Topaz, and
ModelAngelo, several of which are already integrated.

\subsection{Limitations}

CryoProcess currently wraps RELION and a small number of external tools; integration with other
cryo-EM packages (e.g., cryoSPARC, cisTEM) would require implementing additional builder classes.
The platform does not implement its own image processing algorithms, so it cannot be used
independently of RELION. The web-based approach, while lowering the barrier to entry, may not
suit users who prefer scripted, reproducible workflows---though the full RELION command for each
job is recorded and displayed in the interface for reference.

\subsection{Future Directions}

Planned developments include integration of machine learning-based particle quality scoring
(e.g., CryoRanker \citep{Yan2025}) as an automated filtering stage between particle extraction
and classification, further reducing the need for manual 2D class selection.
Additional planned features include support for heterogeneous refinement workflows, integration
with the Electron Microscopy Public Image Archive (EMPIAR) for direct data import, and a REST
API for programmatic pipeline construction.

% ============================================================
% AVAILABILITY
% ============================================================
\section{Availability}

CryoProcess is open-source software distributed under the MIT license. Source code, documentation,
and installation instructions are available at \url{https://github.com/krpothula/cryoprocess}.
The platform requires Node.js $\geq$18, MongoDB $\geq$6.0, and RELION $\geq$4.0. A SLURM cluster
scheduler is recommended for GPU-accelerated processing stages but is not required for
local execution.

% ============================================================
% ACKNOWLEDGMENTS
% ============================================================
\section{Acknowledgments}

% EDIT: Add funding sources, institutional support, user feedback acknowledgments

% ============================================================
% REFERENCES
% ============================================================
\bibliographystyle{unsrtnat}

\begin{thebibliography}{99}

\bibitem[Kuhlbrandt(2014)]{Kuhlbrandt2014}
Kuhlbrandt, W.
\newblock The resolution revolution.
\newblock \emph{Science}, 343(6178):1443--1444, 2014.

\bibitem[Nogales(2016)]{Nogales2016}
Nogales, E.
\newblock The development of cryo-EM into a mainstream structural biology technique.
\newblock \emph{Nature Methods}, 13(1):24--27, 2016.

\bibitem[Cheng(2015)]{Cheng2015}
Cheng, Y.
\newblock Single-particle cryo-EM at crystallographic resolution.
\newblock \emph{Cell}, 161(3):450--457, 2015.

\bibitem[Bai et~al.(2015)]{Bai2015}
Bai, X.-C., McMullan, G., and Scheres, S.~H.~W.
\newblock How cryo-EM is revolutionizing structural biology.
\newblock \emph{Trends in Biochemical Sciences}, 40(1):49--57, 2015.

\bibitem[Scheres(2012)]{Scheres2012}
Scheres, S.~H.~W.
\newblock RELION: implementation of a Bayesian approach to cryo-EM structure determination.
\newblock \emph{Journal of Structural Biology}, 180(3):519--530, 2012.

\bibitem[Kimanius et~al.(2021)]{Kimanius2021}
Kimanius, D., Dong, L., Sharov, G., Nakane, T., and Scheres, S.~H.~W.
\newblock New tools for automated cryo-EM single-particle analysis in RELION-4.0.
\newblock \emph{Biochemical Journal}, 478(24):4169--4185, 2021.

\bibitem[Punjani et~al.(2017)]{Punjani2017}
Punjani, A., Rubinstein, J.~L., Fleet, D.~J., and Brubaker, M.~A.
\newblock cryoSPARC: algorithms for rapid unsupervised cryo-EM structure determination.
\newblock \emph{Nature Methods}, 14(3):290--296, 2017.

\bibitem[de~la~Rosa-Trev\'{i}n et~al.(2016)]{delaRosa2016}
de~la~Rosa-Trev\'{i}n, J.~M., et~al.
\newblock Scipion: A software framework toward integration, reproducibility and validation in
3D electron microscopy.
\newblock \emph{Journal of Structural Biology}, 195(1):93--99, 2016.

\bibitem[Cianfrocco et~al.(2017)]{Cianfrocco2017}
Cianfrocco, M.~A., Wong-Barnum, M., Youn, C., Wagner, R., and Leschziner, A.
\newblock COSMIC$^2$: A science gateway for cryo-electron microscopy structure determination.
\newblock In \emph{Proceedings of the Practice and Experience in Advanced Research Computing},
pages 22:1--22:5, 2017.

\bibitem[Sehnal et~al.(2021)]{Sehnal2021}
Sehnal, D., et~al.
\newblock Mol*: towards a common library and tools for web molecular graphics.
\newblock \emph{Nucleic Acids Research}, 49(W1):W431--W437, 2021.

\bibitem[Bouvette et~al.(2022)]{Bouvette2022}
Bouvette, J., et~al.
\newblock Automated systematic evaluation of cryo-EM specimens with SmartScope.
\newblock \emph{eLife}, 11:e80047, 2022.

\bibitem[Yan et~al.(2025)]{Yan2025}
Yan, Y., Fan, S., Yuan, F., and Shen, H.
\newblock A comprehensive foundation model for cryo-EM image processing.
\newblock \emph{Nature Methods}, 2025.

\end{thebibliography}

% ============================================================
% FIGURE LEGENDS
% ============================================================
\newpage
\section*{Figure Legends}

\textbf{Figure 1. CryoProcess system architecture.}
(a) High-level architecture showing the three-tier design: React frontend communicating with
the Node.js backend via REST API and WebSocket connections, backed by MongoDB for persistent
storage and interfacing with the SLURM cluster scheduler for job execution. The backend can
connect to SLURM either locally or via SSH to remote clusters, requiring only a shared
filesystem between the two.
(b) Data flow for job submission: user fills parameter form $\to$ Joi schema validation $\to$
stage-specific command builder generates RELION command $\to$ SLURM script creation and
\texttt{sbatch} submission $\to$ status monitoring via polling $\to$ WebSocket broadcast to
connected clients.

\textbf{Figure 2. CryoProcess user interface.}
(a) Project dashboard showing active projects with job counts, creation dates, and quick-access
controls. (b) 2D classification parameter form with validated inputs and smart defaults
inherited from upstream jobs. (c) 2D class average gallery with interactive selection for
downstream processing. (d) Auto-refine dashboard showing Fourier Shell Correlation (FSC)
curve with automatically determined resolution at the 0.143 threshold. (e) Interactive pipeline
tree visualization rendering the complete processing DAG with color-coded job status.
(f) Mol* 3D density map visualization integrated into the post-processing dashboard.

\textbf{Figure 3. Live processing pipeline.}
Schematic of automated live processing workflow. A file watcher monitors the movie acquisition
directory and triggers successive processing stages (Import $\to$ Motion Correction $\to$ CTF
Estimation $\to$ Auto-Picking $\to$ Extraction $\to$ 2D Classification) for each batch of newly
arrived movies. Pipeline statistics are propagated between stages and accumulated across passes.

\end{document}
